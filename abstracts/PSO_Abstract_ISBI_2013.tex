\documentclass[9pt,conference,a4paper]{IEEEtran}
\IEEEoverridecommandlockouts

%\usepackage{...}


\title{Particle Swarm Optimization Multitensor Fitting.}
\author{
	\IEEEauthorblockN{
		Michael Paquette\IEEEauthorrefmark{1},
		Eleftherios Garyfallidis\IEEEauthorrefmark{1},
		Samuel St-Jean\IEEEauthorrefmark{1},
		Pierrick Coup\'e\IEEEauthorrefmark{2},
		Maxime Descoteaux\IEEEauthorrefmark{1}
	}

	\IEEEauthorblockA{\IEEEauthorrefmark{1} Universit\'e de Sherbrooke, Sherbrooke, Canada}
	\IEEEauthorblockA{\IEEEauthorrefmark{2} McGill University, Montreal, Canada}

	\thanks{This work is supported by...}
}




\begin{document}
\maketitle

For the purpose of the ISBI HARDI reconstruction challenge 2013 and for the categories DTI and HARDI we reconstructed the data by fitting a multitensor (MT) with particle swarm optimization (PSO) \cite{?pso?}. 

The goal is to find the optimal parameters so that the MT model, $\sum_{i=0}^N f_i e^{-b g^t D_i g}$, fits the measured signal where $D_i$ is a rank 2 symmetric tensor with volume fraction $f_i$, $g$ the normalized gradient wavevector and $b$ the corresponding b-value. This is accomplished by minimizing the squared error $\sum_{k=0}^M \left( \sum_{i=0}^N f_i e^{-b_k g_k^t D_i g_k} - y_k \right) $ for a fixed diffusion signal $y = \{y_k\}_{k=0}^M$, number of compartement $N$ and gradient scheme $ \{b_k, g_k\}_{k=0}^M$.

This minimization is perform using the pso. The pso is a stochastic obtimization algorithm using population interaction to search the parameter space. We initialize P particules (points in $\mathbf{P}^n$, the n parameters space of real value) and move them around according to a velocity $P_z^{(t+1)} = P_z^{(t)} + \phi_1 * u_1 * P_z^{(t)} + \phi_2 * u_2 * P_z^{best} + \phi_3 * u_3 * P_{best}^{best}$ where $u_l ~ \mathcal{U}[0,1]$ and $\phi_l$ are user inputed parameters affecting the pso's behavior.

Considering the given ground truth, a binary conectivity matrix with given ROI, we aimed at validating the quality of our method on something comparable. We computed a tractography for all different parameters combinason of local model. We computed a "connectivity" matrix for the ROIs from the track (track count) and normalized it with the ROI's size (we had seeded only from the ROIs and had a fixed number of seeds per voxel). We then  applied different threshold to obtain binary connectivity matrix and count the direct error (# false bundle + # missing bundle), using that information with several threshold to get a good grasp of the connectivity strenght. Indeed, a good tracking should possess a large range of value with low direct error, meaning that, as the threshold grow, false bundle disappear faster than true bundle and as the threshold goes down, missing bundle appear faster than new false bundle.

The DW dataset was denoised with the adaptive nonlocal means \cite{manjon-coupe:10}
using a rician noise model. As proposed in \cite{descoteaux-wiest-daessle-etal:08}, each DW images were processed independently.



Hello \cite{manjon-coupe:10}, \cite{descoteaux-deriche-etal:09}, \cite{Descoteaux2008}, \cite{tournier-calamante-etal:07}.

...

...

...

...

...

...

...

...

...

...

...

...

...

...

...

...

...

...

...

...

...

...

...

...

...

...

...

...

...

...

...

...

...

...

...

...

...

...

...


...

...

...

...

...

...

...

...

...

...

...

...



...

...

...

...

...

...

...

...

...

...

...

...

...

...

...

...

...

...

...

...

...

...

...

...

...

...

...

...

...

...

...

...

...

...

...

...

...

...

...

...


\bibliographystyle{ieeetr}
\bibliography{/home/eleftherios/Documents/scil-bibtex/scilBibTex}

\end{document}


